% enable this to activate the version for PRINT
% => asymmetric alternating left/right margins
% \newcommand*{\printversion}{}%

%% | ---------------- document meta information --------------- |

\newcommand{\Author}{Ing. John Smith}
\newcommand{\Department}{Department of Cybernetics}
\newcommand{\Supervisor}{Ing. My Supervisor, Ph.D.}
\newcommand{\SupervisorSpecialist}{Ing. My Specialist, Ph.D.}
\newcommand{\Programme}{Insert My Study Programme Name}
\newcommand{\Field}{Insert My Study Subprogramme Name}
\newcommand{\Title}{My Thesis Title\\[0.5em]can Span Multiple Lines}
\newcommand{\DocName}{Bachelor Thesis / Master Thesis / Doctoral Thesis}
\newcommand{\Keywords}{Unmanned Aerial Vehicles, Automatic Control}
\newcommand{\KlicovaSlova}{Bezpilotní Prostředky, Automatické Řízení}
\newcommand{\Year}{2021}
\newcommand{\Month}{January}
\newcommand{\Date}{\Month~\Year}
\newcommand{\Location}{Prague}

%!TEX root = ../main.tex

%% | ----------------------- page setup ----------------------- |

% select the page format based on the print / !print version
\pdfoutput=1
\ifdefined\printversion
  \documentclass[a4paper,11pt,twoside,openright]{book}
\else
  \documentclass[a4paper,11pt,oneside]{book}
\fi

% when !print, then no white pages
\newcommand{\conditionalClearPage}{
  \ifdefined\printversion
    \cleardoublepage
  \else
    \clearpage
  \fi
}

% print version has different margins to accommodate the spine of the book
\ifdefined\printversion
  \usepackage[a4paper,margin=3.2cm,inner=3.4cm,outer=2.0cm]{geometry}
\else
  \usepackage[a4paper,margin=3.2cm,inner=2.7cm,outer=2.7cm]{geometry}
\fi

%% | ----------------- commonly used packages ----------------- |

\usepackage[english]{babel}
\usepackage[utf8]{inputenc}
\usepackage{csquotes}
\usepackage{amsmath,amsfonts,amssymb,bm}
\usepackage{nicefrac}
\usepackage{algorithm,algpseudocode}
\usepackage[title,titletoc]{appendix}
\usepackage{latexsym}
\usepackage{a4wide}
\usepackage{color}
\usepackage{indentfirst}
\usepackage{graphicx}
\usepackage{fancyhdr,lastpage}
\usepackage{longtable}
\usepackage{pifont}
\usepackage{makeidx}
\usepackage{multirow}
\usepackage{dcolumn}
\usepackage{epstopdf}
\usepackage{url}
\usepackage{listings}
\usepackage{relsize}
\usepackage{pdfpages}
\usepackage{url}
\usepackage{lipsum}
\usepackage{isotope}
\usepackage{verbatim}
\usepackage{xcolor}
\usepackage{tcolorbox}
\usepackage[hidelinks]{hyperref}
\usepackage{multicol}
\usepackage{subfig}
\usepackage[export]{adjustbox}

\hyphenation{}

\usepackage[printonlyused]{acronym}

%% | ------------ siunitx for units of measurements ----------- |

\usepackage{siunitx}
\DeclareSIUnit \parsec {pc}
\DeclareSIUnit \electronvolt {eV}
\DeclareSIUnit \pixel {px}
\DeclareSIUnit \arcmin {arcmin}
\DeclareSIUnit \erg {erg}
\DeclareSIUnit \joul {J}

%% | --------------- change formatting of lists --------------- |

\usepackage{enumitem}
\setlist{nosep}

%% | -------------------- table of contents ------------------- |

\usepackage[subfigure]{tocloft}

\tocloftpagestyle{plain}

%% | ----------------- formatting of a chapter ---------------- |

\usepackage{titlesec}
\titleformat{\chapter}[block]
{\normalfont\huge\bfseries}{Chapter \thechapter\\\vspace{0.1em}\\}{1em}{\Huge}
% {?}{before}{after}
\titlespacing*{\chapter}{0pt}{-1em}{2em}

%% | ------------------------ biblatex ------------------------ |

\usepackage[backend=bibtex,defernumbers=true,style=ieee,sorting=ydnt,sortcites=true]{biblatex}

% define the source file with bibliography
\addbibresource{main.bib}

\renewcommand*{\bibfont}{\Font}

% add suffix "a" to publications containing the keyword "mine"
% add suffix "c" to publications containing the keyword "mine" && "core"
\DeclareFieldFormat{labelnumber}{%
  \ifkeyword{mine}
    {\ifkeyword{core}
      {{\number\numexpr#1}c}%
      {{\number\numexpr#1}a}%
    }%
    {#1}%
}

\DeclareCiteCommand{\tabcite}%[\mkbibbrackets]
  {\usebibmacro{cite:init}%
   \usebibmacro{prenote}}
  {\usebibmacro{citeindex}%
   \usebibmacro{cite:comp}}
  {}
  {\usebibmacro{cite:dump}%
   \usebibmacro{postnote}}

%%}

%% | ---------------------- custom macros --------------------- |

\newcommand{\strong}[1]{\textbf{#1}}
\newcommand{\coord}[1]{\textbf{#1}}
\newcommand{\norm}[1]{\left\lvert#1\right\rvert}
\newcommand{\m}[1]{\ensuremath{\mathbf{#1}}}
\newcommand\numberthis{\addtocounter{equation}{1}\tag{\theequation}}
\newcommand{\add}[1]{{\color{green} {#1}}}
\newcommand{\todo}[1]{{\color{red} TODO {#1}}}
\newcommand{\updated}[1]{{\color{blue} {#1}}}
\newcommand{\reffig}[1]{Fig.~\ref{#1}}
\newcommand{\refalg}[1]{Alg.~\ref{#1}}
\newcommand{\refsec}[1]{Sec.~\ref{#1}}
\newcommand{\reftab}[1]{Table~\ref{#1}}
\newcommand{\real}{\mathbb{R}}
\newcommand{\red}[1]{{\color{red} #1}}
\newcommand{\minus}{\scalebox{0.75}[1.0]{$-$}}
\newcommand{\plus}{\scalebox{0.8}[0.8]{$+$}}
\newcommand{\figvspace}{\vspace{-1em}}

%% | --------------- change listings and itemize -------------- |

\lstset{breaklines=true,captionpos=b,frame=single,language=sh,float=h}
\lstloadlanguages{sh,c}
\def\lstlistingname{Listing}
\def\lstlistlistingname{Listings}

%% | -------------------- layout parameters ------------------- |

% TODO what is this?
\def\clinks{false}

% no indent, free space between paragraphs
\setlength{\parindent}{1cm}
\setlength{\parskip}{1ex plus 0.5ex minus 0.2ex}

% offsets the head down
\setlength{\headheight}{18pt}

% foot line
\renewcommand{\footrulewidth}{0.4pt}

%% | -------------- define the 'full' page style -------------- |

\fancypagestyle{full}{%

  % clear the default layout
  \fancyhead{}
  \fancyfoot{}

  % page header
  \fancyhead[LO]{\leftmark}
  \fancyhead[RE]{\rightmark}
  \fancyhead[LE,RO]{\thepage/\pageref{LastPage}}

  % page footer
  \fancyfoot[L]{CTU in Prague}
  \fancyfoot[R]{\Department}
  \fancyfoot[C]{}

}

%% | -------------- define the 'plain' page style ------------- |

\fancypagestyle{plain}{%

  % clear the default layout
  \fancyhead{}
  \fancyfoot{}

  % page header
  \fancyhead[LE,RO]{\thepage}

}

%% | -------------- Adjust style of chapter names ------------- |

\renewcommand{\chaptermark}[1]{\markboth{\MakeUppercase{\thechapter.\ #1}}{}}

%% | -------- European layout, no extra space after '.' ------- |

\frenchspacing

%% | ----------- adjust the style of the first page ----------- |

\makeatletter
\renewcommand\chapter{\if@openright\cleardoublepage\else\clearpage\fi
                    \thispagestyle{full}% original style: plain
                    \global\@topnum\z@
                    \@afterindentfalse
                    \secdef\@chapter\@schapter}
\makeatother

%% | -------------------------- todos ------------------------- |

% without this it does not compile!
\let\bibfont\small

% what does this do?
\def\changemargin#1#2{\list{}{\rightmargin#2\leftmargin#1}\item[]}
\let\endchangemargin=\endlist


\begin{document}

%!TEX root = ../main.tex

\begin{titlepage}
  \begin{center}

    \textsc{\Large Czech Technical University in Prague}\\[1em]
    \textsc{\large Faculty of Electrical Engineering\\
    \Department\\
    Multi-robot Systems\\[3em]
    }
    \includegraphics[height=4.1cm]{fig/ctu_lion.pdf}\\[3em]

    \textbf{\textsc{\Huge \Title}}\\[2em]

    \textbf{\Large Master's Thesis}\\[6em]

    \textbf{\huge \Author}\\[6em]

    {\large \Location, \Date}\\[3em]

    Study programme: \Programme\\
    Branch of study: \Field\\[4em]

    \textbf{Supervisor: \Supervisor}\\

    \vspace{2pt}

  \end{center}
\end{titlepage}


\pagestyle{plain}

\conditionalClearPage

\pagenumbering{roman}
%!TEX root = ../main.tex

\section*{Acknowledgments}

Firstly, I would like to express my gratitude to my supervisor.

\vspace{2.5cm}


\conditionalClearPage
%!TEX root = ../main.tex

\begin{changemargin}{0.8cm}{0.8cm}

~\vfill{}

\section*{Copyright}
\vskip 0.5em

This thesis is a compilation of several articles published and submitted during my Ph.D. studies.
The included publications are presented under the copyrights of IEEE, and Springer for posting the works for internal institutional uses.
The works are protected by the copyrights of respective publishers and can not be further reprinted without the publishers' permission.

\vskip 1.0cm

\textsuperscript{\textcopyright} IEEE, 2020\\
\textsuperscript{\textcopyright} Springer, 2020\\

\end{changemargin}


\conditionalClearPage
%!TEX root = ../main.tex

\begin{changemargin}{0.8cm}{0.8cm}

~\vfill{}

\section*{Abstract}
\vskip 0.5em

The study of autonomous \acp{UAV} has become a prominent sub-field of mobile robotics.

\vskip 1em

{\bf Keywords} \Keywords

\vskip 2.5cm

\end{changemargin}


\conditionalClearPage
%!TEX root = ../main.tex

\begin{changemargin}{0.8cm}{0.8cm}

~\vfill{}

\section*{Abstrakt}
\vskip 0.5em

\sloppy
Výzkum na poli autonomních bezpilotních prostředků (UAV) se stal významným oborem mobilní robotiky.

\vskip 1em

{\bf Klíčová slova} \KlicovaSlova

\vskip 2.5cm

\end{changemargin}


\conditionalClearPage

\begin{changemargin}{0.8cm}{0.8cm}

~\vfill{}

\section*{Abbreviations}
../../common_resources/src/abbreviations.tex

\vskip 2.5cm

\end{changemargin}

\conditionalClearPage
\tableofcontents

\conditionalClearPage

\pagestyle{full}

%% --------------------------------------------------------------
%% |                        introduction                        |
%% --------------------------------------------------------------

\chapter{Introduction}

First, introduce the reader to the research topic.
Start with the most general view and slowly converge to the particular field, subfield, and the challenges you face.
You can cite others work here \cite{baca2020mrs}.

\pagenumbering{arabic}

\section{Related works}

This section should contain related state-of-the-art works and their relation to the author's work.
We usually cite the original works like this \cite{benallegue2008high}.
You can also cite multiple papers at once like this \cite{baca2016embedded, baca2020mrs}.

\section{Contributions}

This section should briefly describe the author's contributions to his field of research.

%% --------------------------------------------------------------
%% |                How to write thesis in LaTeX                |
%% --------------------------------------------------------------

\chapter{How to write thesis in LaTeX}

\section{Mathematical notation with LaTeX}

\section{Abbreviations with Acronym}

Abbreviations are handled by the \emph{acronym} package.
Example sentence with abbreviations: ``\ac{UAV} is a flying vehicle that commonly uses \ac{LiDAR} and \ac{GPS} receiver''.
Please, read the documentation (\url{http://mirrors.ctan.org/macros/latex/contrib/acronym/acronym.pdf}).

\section{Units of measurements with Siunitx}

Typesetting units has never been easier with the Siunitx package.
Acceleration is measure in \si{\meter\per\second\squared}.
Gravity accelerates objects at a rate $\approx \SI{9.81}{\meter\per\second\squared}$ near the sea level.
You can define your own units if you want.

\section{2D Diagrams with Tikz}

\section{Data plots with PGFPlots}

\begin{itemize}
  \item Documentation and manual: \url{https://ctan.org/pkg/pgfplots}
  \item Compile the plots individually and then include the pdfs, because it can take a long time.
  \item Example located in \texttt{fig/plots/example\_plot}, see \reffig{fig:pgfplots}.
  \item You could include the latex file directly, however, it will take longer time to compile and platforms such as Overleaf can have a problem with that.
\end{itemize}

\begin{figure}[!h]
  \centering
  \includegraphics[width=1.0\textwidth]{./fig/plots/example_plot/hypotheses.pdf}
  \caption{Example of a 2D plot using \emph{PGFPlots}.}
  \label{fig:pgfplots}
\end{figure}

\section{3D Plots with Sketch}

See the example in \reffig{fig:coordinate_systems}.

\begin{itemize}
  \item Documentation and manual: \url{http://www.frontiernet.net/~eugene.ressler/}
  \item Cross-compilation from \emph{Sketch} to \emph{pdf} using the \texttt{fig/sketch/compile\_sketch.sh} script.
\end{itemize}

\begin{figure}[!h]
  \centering
  \includegraphics[width=0.4\textwidth]{./fig/sketch/coordinate_frames.pdf}
  \caption{Depiction of the used coordinate systems. The image was drawn using \emph{Sketch}.}
  \label{fig:coordinate_systems}
\end{figure}

\section{Image collages with Subfig}

\section{Citations with Biblatex}

\section{Image overlays with Tikz}

%% --------------------------------------------------------------
%% |                         Conclusion                         |
%% --------------------------------------------------------------

\chapter{Conclusion\label{chap:conclusion}}

Briefly summarize the achieved results.

%% --------------------------------------------------------------
%% |                         Appendices                         |
%% --------------------------------------------------------------

\appendix
\renewcommand\chaptername{Appendix}

\chapter{References}

\printbibliography[heading=none,title={}]

% \appendix
\renewcommand{\thechapter}{B}
\renewcommand\chaptername{Appendix B}

\chapter{Appendix B}

\end{document}
